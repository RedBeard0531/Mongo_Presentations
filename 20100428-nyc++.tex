\documentclass{beamer}

\usetheme{Warsaw}
\setbeamerfont*{frametitle}{size=\normalsize,series=\bfseries}
\setbeamertemplate{navigation symbols}{}
%\usefonttheme{professionalfonts}

\usepackage{listings,bera}
\usepackage[english]{babel}
\usepackage[utf8x]{inputenc}
\usepackage{times}
\usepackage[T1]{fontenc}
\usepackage{ulem}
\usepackage{listings}
\usepackage{textcomp}
\usepackage{graphicx}
\usepackage{subfigure}
\usepackage{hyperref}

%\definecolor{fore}{RGB}{249,242,215}
%\definecolor{back}{RGB}{51,51,51}
%\definecolor{title}{RGB}{255,0,90}
%\setbeamercolor{titlelike}{fg=title}
%\setbeamercolor{normal text}{fg=fore,bg=back}

\definecolor{keywords}{RGB}{255,0,90}
\definecolor{comments}{RGB}{60,179,113}
\definecolor{strings}{RGB}{60,179,60}
\definecolor{numbers}{RGB}{179,60,60}
\lstset{language=C++,
        extendedchars=false,
        keywordstyle=\color{keywords},
        commentstyle=\color{comments},
        stringstyle=\color{strings},
        escapechar=@
        }

% adds the \MongoLogo command to put the logo on a slide
\usepackage[absolute,overlay]{textpos}
\setlength{\TPHorizModule}{1mm}
\setlength{\TPVertModule}{1mm}
\newcommand{\MongoLogo}{
\begin{textblock}{14}(2.0,0.7)
  %\includegraphics[height=0.8cm]{logo-mongo-ondark.png}
\end{textblock}
}

\pgfdeclareimage[height=2in]{featuregraph}{featuresPerformance.png}
\pgfdeclareimage[height=2.5in]{sharding}{sharding.png}

\title{Dynamic C++}
\subtitle{Or how to write python code in C++}
\author{Mathias Stearn}
\institute{ \includegraphics[height=0.8cm]{10gen.png} }
\date{NYC++ Meetup -- April 28, 2010}

\AtBeginSection[]
{
  \begin{frame}<beamer>{}
    \MongoLogo
    \tableofcontents[currentsection]
  \end{frame}
}


\begin{document}

\begin{frame}
  \titlepage
\end{frame}

\begin{frame}
  \MongoLogo
  \tableofcontents
\end{frame}

\section{boost::any}

\begin{frame}{What is boost::any}
  \MongoLogo
  \begin{itemize}
    \item A discriminated, type-safe union of all types
      \begin{itemize}
        \item read: a better void*
      \end{itemize}
    \item You can assign anything to it
    \item It knows what you put put in
    \item You get out what you put in
      \begin{itemize}
        \item No conversions between types
        \item Not even numeric types
      \end{itemize}
    \item Value semantics (copies actually copy)
  \end{itemize}
\end{frame}

\begin{frame}[fragile] {Basic usage}
  \MongoLogo
  \begin{lstlisting}
boost::any a(1);
a = 1.0;
a = "1";
a = string("1");

any_cast<string>(a).c_str();
any_cast<string&>(a) = "hello";

any_cast<int>(a); //throws bad_any_cast
any_cast<int>(&a); //return null

a.type() == typeid(string);
a.type() != typeid(int);
  \end{lstlisting}
\end{frame}

\begin{frame}[fragile] {Usage in collection}
  \MongoLogo

  \begin{lstlisting}
typedef std::vector<boost::any> many

many array;
array.push_back(42);
array.push_back("Mathias");
  \end{lstlisting}

  \pause

  \begin{lstlisting}
BOOST_FOREACH(const any& item, many){
    if (int* i = any_cast<int>(&item))
        cout << *i << endl;
    else if (string* s = any_cast<string>(&item))
        cout << *s << endl;
    else
        assert(!"unrecognized type");
}
  \end{lstlisting}

\end{frame}

\begin{frame}[fragile] {Usage in collection cont.}
  \MongoLogo

  \begin{lstlisting}
typedef std::vector<boost::any> many

many slice(many array, int skip, int limit){
    many out;
    BOOST_FOREACH(any& item, many){
      if (skip) { skip--; continue }
      if (limit-- == 0) break;

      out.push_back(item);
    }
    return out;
}
  \end{lstlisting}

\end{frame}

\begin{frame}[fragile] {Virtual Methods}
  \MongoLogo

  \begin{lstlisting}
class Example{
    // This is illegal!
    template <typename T>
    virutal void doSomething(const T& anything);

    // This is ok
    virutal void doSomething(boost::any anything);
}
  \end{lstlisting}
\end{frame}

\section{boost::variant}
\begin{frame}{What is boost::variant}
  \MongoLogo
  \begin{itemize}
    \item A discriminated, type-safe union of selected types
      \begin{itemize}
        \item read: Haskell/ML types in C++
      \end{itemize}
    \item You declare the supported types
    \item Stack-based storage
    \item Uses visitor pattern for access
    \item Feels ``cleaner'' than boost::any
  \end{itemize}
\end{frame}

\begin{frame}[fragile] {Basic usage}
  \MongoLogo
  \begin{lstlisting}
    typedef boost::variant<int, double> number;
    number n = 1;
    n = 1.0;
    
    cout << n << endl; // built-in

    get<double>(n);
    get<int>(n); // throws bad_get
  \end{lstlisting}
\end{frame}

\begin{frame}[fragile] {Visitors}
  \MongoLogo
  \begin{lstlisting}
struct Square : boost::static_visitor<>{
  void operator()(int& i) { i*= i; }
  void operator()(double& d) { d *= d; }
} square;

number n = 10;
apply_visitor(square, n);
  \end{lstlisting}
\end{frame}

\begin{frame}[fragile] {Nicer Visitors}
  \MongoLogo
  \begin{lstlisting}
struct Square : boost::static_visitor<>{
  void operator()(int& i) const { i*= i; }
  void operator()(double& d) const { d *= d; }
  void operator()(number& n) const {
    apply_visitor(Square(), n);
  }
} square;

number n = 10;
square(n);
  \end{lstlisting}
\end{frame}

\section{BSON}
\begin{frame}{What is BSON}
  \MongoLogo
  \begin{itemize}
    \item A binary JSON format used in MongoDB
    \item Supports nested objects and arrays
    \item Supports a fixed set of types
    \item Objects are ``frozen'' once created
    \item Nice C++ library
  \end{itemize}
\end{frame}

\begin{frame}[fragile] {Basic usage}
  \MongoLogo
  \small
  \begin{lstlisting}
BSONObj obj =
    BSON( "name" << BSON( "first" << "Mathias"
                      << "last" << "Stearn")
       << "company" << "10gen"
       << "languages" << BSON_ARRAY( "C++"
                                  << "Python");
       << "minions" << 0
       )

obj["name"].type() == Object;
obj["name"]["first"].type() == String;
obj["minions"].type() == Int;

BSONObj nameobj = obj["name"].embeddedObject();
string fullname = name["first"].String() + " "
                + name["last"].String();
                        
  \end{lstlisting}
\end{frame}

\begin{frame}[fragile] {Basic usage}
  \MongoLogo
  \huge
  Questions?
\end{frame}


\end{document}

% vim: set softtabstop=2
